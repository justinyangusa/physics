\documentclass[12pt,letterpaper]{article}
\usepackage{amsmath}
\usepackage{gensymb}
\usepackage{textcomp}
\usepackage[margin=0.7in]{geometry}
\setlength{\parindent}{0pt}
\title{Physics Club Handout 2: Newton's Laws}
\setcounter{page}{2}
\setcounter{section}{1}
\begin{document}
\section{Physics Club Handout: Newton's Laws}
For all these problems, assume acceleration due to gravity is 9.81 m\slash s$^2$.

\paragraph{Beginner problems:}
\begin{enumerate}
\item
A 50 kg human in freefall experiences a force from air resistance equal to $\vec{F}_{air} = -\vec{v}^{\,2}*0.3141\frac{\text{kg}}{\text{m}}$, where $\vec{v}$ is the person's instantaneous velocity. Find the human's terminal velocity.

\item
A hawk flies in a horizontal circle of radius 12 m at a constant velocity of 4 m\slash s. Find its centripital acceleration under these conditions.
The hawk then increases its speed at a rate of 1.2 m\slash s$^2$; find its new acceleration (and the direction of its acceleration).

\item
A 40.0-kg child swings in a swing supported by two chains, each 3.00 m long. The tension in each chain at the lowest point is 350 N. Find the child's speed at the lowest point and the maximum height to which the child rises.
\end{enumerate}

\paragraph{Intermediate problems:}
\begin{enumerate}
\setcounter{enumi}{3}
\item
An object moving through a fluid (at sufficiently high velocity such that viscous forces are insignificant) experiences a retarding force of $\vec{F}_{drag} = -\frac{1}{2}\vec{v}^{\,2}*C$, where $\vec{v}$ is the object's instantaneous velocity, and $C$ is some constant based on the fluid's density and the object's shape.

Given that an object of mass $m$ is dropped in air at time $t = 0$, find the velocity of the object at any given time in terms of $m$, $t$, and $C$.

\item
An amusement park ride is set up as a giant swing that starts at an angle of $80\degree$ to the vertical, and allows the swing to fall freely. For legal reasons, the maximum g-force a rider can experience is $5g$'s (where $g$ = 9.81 m\slash s$^2$). Assuming no air resistance, what is the largest they can make the swing and still avoid litigation?

\item
A plumb bob (a weight hanging from a string) usually does not hang perfectly vertically (i.e. along a line directed towards the center of the earth). By how much does a plumb bob deviate from vertical here in Palo Alto (latitude of 37.4\degree\,N), assuming the earth is spherical and has radius 6380 km?
\end{enumerate}

\paragraph{Advanced problems:}
\begin{enumerate}
\setcounter{enumi}{6}
\item
An object moving througha a fluid experiences a force $\vec{F}_{drag} = -(ar\vec{v} + br^2\vec{v}^2)$ exerted on a sphere of radius $r$ moving through a fluid at speed $v$, where $a$ and $b$ are constants based on the shape of the object and the surrounding atmosphere. For spherical objects in air at sea level, $a = 3.10 \times 10^{-4}\,\text{Pa}\cdot \text{s}$ and $b = 0.870$ g\slash L.
Find the velocity of a water droplet of 100 \textmu m freefalling at time $t$, where $t$ is the time elapsed since it was released from rest.

\item
A 1 kg block is sitting on a table -- the coefficient of static friction between the toaster and the table is 0.4. A string is attached to the block. You pull on the string to make the block move; at what angle should you pull the string to minimize the force necessary to move the block, and what is the force?

\item
A time-dependant force $\vec{F} = (8.00\,\hat{i} -4.00\,t\,\hat{j})\,\text{N}$ (where $t$ is in seconds) is exerted on a 2 kg object initially at rest. At what time will it be moving at a speed of 15 m\slash s? What is the total displacement and distance of the object at this time?
\end{enumerate}

\end{document}