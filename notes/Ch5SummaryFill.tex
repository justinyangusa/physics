\documentclass[11pt,a4paper]{report}
\usepackage{asymptote}
\usepackage{wrapfig}
\addtolength{\oddsidemargin}{-.875in}
\addtolength{\evensidemargin}{-.875in}
\addtolength{\textwidth}{1.75in}
\addtolength{\topmargin}{-.875in}
\addtolength{\textheight}{1.75in}
\begin{document}
\pagestyle{empty}
\setcounter{secnumdepth}{0}

\begin{center}
\Large{CHAPTER 5 SUMMARY. \textbf{Energy}}

\large{Justin Yang}

\sc{November 22, 2012}
\end{center}

\begin{wrapfigure}{r}{0.2\textwidth}
\vspace{-20pt}
\begin{center}
\includegraphics[width=0.18\textwidth]{images/05_!1work.JPG}
\end{center}
\end{wrapfigure}
\section{(1)\underline{\hspace{3cm}}}
The (2)\underline{\hspace{3cm}} done by a constant force $\vec{F}$ that moves an object a displacement $\Delta{x}$ is defined as $$\left(3\right)\underline{\hspace{5cm}}.$$ So $W = \vec{F} \cdot \Delta{\vec{x}}$.
\\Work is a (4)\underline{\hspace{2cm}}. The SI unit of work is the (5)\underline{\hspace{2cm}} (J), $1 \mathrm{\ J} = 1 \mathrm{\ N} \cdot \mathrm{m} = 1 \mathrm{\ kg} \cdot \mathrm{m}^2 / \mathrm{s}^2$.

\smallskip

\begin{wrapfigure}{r}{0.2\textwidth}
\vspace{-20pt}
\begin{center}
\includegraphics[width=0.18\textwidth]{images/05_!3varying.JPG}
\end{center}
\end{wrapfigure}

\noindent
A (6)\underline{\hspace{2cm}} is any object where all of its parts undergo equal $\Delta{x}$ over any $\Delta{t}$. The total work done on a particle is the same as the work done by the net force on the particle, so the work done is the area under the $F_x$-versus-$x$ curve: $$\left(7\right)\underline{\hspace{5cm}}.$$

\section{(8)\underline{\hspace{4cm}}}
Under a constant \textit{net} force $F_\mathrm{net}$ acting along a straight line on a particle of mass $m$, which is displaced by $\Delta{x}$ along the straight line, the work done on the particle is $$\left(9\right)\underline{\hspace{5cm}}.$$
Applying Netwon's second law (10)\underline{\hspace{2cm}} and the kinematic relation (11)\underline{\hspace{2cm}}, we have $$\left(12\right)\underline{\hspace{5cm}}.$$
The quantity $\frac{1}{2} mv^2$ is defined as the (13)\underline{\hspace{3cm}} of the particle $$\left(14\right)\underline{\hspace{5cm}}.$$
\\Kinetic energy is a (15)\underline{\hspace{2cm}}. The SI unit of kinetic energy is the same as work: $\mathrm{kg} \cdot \mathrm{m}^2 / \mathrm{s}^2$ or J.
\\Kinetic energy depends on the mass and speed of the particle but not the direction of motion.

\smallskip

\noindent
$W = \Delta{K}$. This is true even when the force is varying. This is known as the (16)\underline{\hspace{3cm}}.

\section{(17)\underline{\hspace{4cm}}}
The (18)\underline{\hspace{3cm}} of a system is the energy associated with the configuration of the system. Often the work done by external forces on a system may result in an increase in the potential energy of the system.

\begin{wrapfigure}{r}{0.1\textwidth}
\vspace{-23pt}
\begin{center}
\includegraphics[width=0.09\textwidth]{images/05_!4gpe.JPG}
\end{center}
\vspace{-20pt}
\end{wrapfigure}

\smallskip

\noindent
(19)\underline{\hspace{5cm}} The gravitational force between an object of mass $m$ and the Earth is $\vec{F} = -mg\,\hat{j}$, where $h$, $h_0 \ll r_E$, so the work done by gravity is $$\left(20\right)\underline{\hspace{7cm}}.$$
When the object is near the surface of the Earth, the gravitational potential energy $$\left(21\right)\underline{\hspace{3cm}}.$$
Thus, the work done by gravity is at the expense of the gravitational potential energy: $$\left(22\right)\underline{\hspace{3cm}}.$$
(23)\underline{\hspace{5cm}} The work done by the spring force, $F = -kx$, is given as $$\left(24\right)\underline{\hspace{7cm}}.$$
When the spring potential energy is zero at $x = 0$, the spring potential energy can be defined as $$\left(25\right)\underline{\hspace{3cm}}.$$
The work done by the spring force is then at the expense of the spring potential energy $$\left(26\right)\underline{\hspace{3cm}}.$$

\subsection{(27)\underline{\hspace{9cm}}}

\begin{wrapfigure}{r}{0.2\textwidth}
\vspace{-20pt}
\begin{center}
\includegraphics[width=0.18\textwidth]{images/05_!5conservative.JPG}
\end{center}
\vspace{-20pt}
\end{wrapfigure}

A force is conservative if on a particle $W_\mathrm{net} = 0$ around \textit{any} closed path.
\\We can use this property to define a (28)\underline{\hspace{4cm}} $U$ such that the force is the negative of the slope of the potential-energy $U$-versus-$x$ curve: $$\left(29\right)\underline{\hspace{5cm}}.$$
(30)\underline{\hspace{4cm}} are forces that are not conservative.

\section{(31)\underline{\hspace{7cm}}}
A (32)\underline{\hspace{3cm}} is a collection of particles. All forces are either (33)\underline{\hspace{3cm}} or (34)\underline{\hspace{3cm}}. The change in $E_\mathrm{net}$ of a system is done through work and heat. Since $K = \sum{K_i}$, we obtain by the work-energy theorem $$\left(34\right)\underline{\hspace{5cm}}.$$
The work done by all internal conservative forces can be recast as the change in the total potential energy of the system: $$\left(35\right)\underline{\hspace{3cm}}.$$
The sum $E_\mathrm{mech} = K + U$ is known as the total mechanical energy of the system, $$\left(36\right)\underline{\hspace{7cm}}.$$
When $W_\mathrm{ext} = 0$ and $W_\mathrm{nc} = 0$, we get the (37)\underline{\hspace{5cm}}: $$\left(38\right)\underline{\hspace{5cm}}.$$

\section{(39)\underline{\hspace{7cm}}}
For an isolated system, we have $W_\mathrm{ext} = 0$ and we may account of $W_\mathrm{nc}$ by changes in forms of energy other than mechanical energy. (40)\underline{\hspace{5cm}}: $$\left(41\right)\underline{\hspace{7cm}}.$$
Work and heat are the ways to transfer energy in or out of a system. When $\Delta{Q} = 0$, we have: $$\left(42\right)\underline{\hspace{9cm}}.$$

\section{(43)\underline{\hspace{5cm}}}
Power is the rate at which energy is transferred. The average (44)\underline{\hspace{3cm}} supplied by a force $\vec{F}$ is the rate at which the force does work: $$\left(45\right)\underline{\hspace{5cm}},$$ $$\left(46\right)\underline{\hspace{5cm}}.$$
The SI unit of power is J/s, also called the (47)\underline{\hspace{3cm}}. $1 \mathrm{\ W} = 1 \mathrm{\ J} / \mathrm{s} = 1 \mathrm{\ kg} \cdot \mathrm{m}^2 / \mathrm{s}^3$.
\end{document}