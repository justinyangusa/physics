\documentclass[11pt,a4paper]{report}
\usepackage{asymptote}
\usepackage{wrapfig}
\addtolength{\oddsidemargin}{-.875in}
\addtolength{\evensidemargin}{-.875in}
\addtolength{\textwidth}{1.75in}
\addtolength{\topmargin}{-.875in}
\addtolength{\textheight}{1.75in}
\begin{document}
\setcounter{secnumdepth}{0}
\begin{center}
\Large{CHAPTER 4 SUMMARY. \textbf{The Laws of Motion}}

\large{Justin Yang}

November 2, 2012
\end{center}

\section{Newton's Laws}
``The discovery of the laws of motion was a dramatic moment in the history of science. Before Newton's time, the motion of the things like the planets were a mystery, but after Newton there was a complete understanding."---Richard Feynman

\hspace{1mm}

\noindent
A \textbf{force} is simply a push or a pull on some object.\\A force has both magnitude and direction, so it is a vector quantity.

\subsection{Newton's First Law}
Galileo made a great advance in the understanding of motion when he discovered the \textit{principle of inertia}: if an object is left alone, is not perturbed, it continues to move with a constant velocity in a straight line if it was originally moving, or it continues to stand still if it was just standing still.
\begin{itemize}
	\item This never appears to be the case in nature: if we slide a block across a table it stops.
	\item It requires a certain imagination to find the right rule.
\end{itemize}
\textbf{Inertia} The tendency of an object to continue in its original state of motion.

\hspace{1mm}

\noindent
\textbf{Mass} A measure of inertia. That is a measure of the object's resistance to changes in its motion due to a force.

\hspace{1mm}

\noindent
Newton formalized Galileo's principle of inertia into \textbf{Newton's first law of motion}:\\\emph{An object moves with a constant velocity unless acted upon by a nonzero net force.}

\hspace{1mm}

\noindent
\textbf{Inertial Reference Frame} Newton's first law allows the definition of inertial reference frame. \emph{If no forces act on an object, any reference frame with respect to which the acceleration of the object remains zero is an inertia reference frame.}

\subsection{Newton's Second Law}
Newton's second law provides a specific way of determining how the velocity of an object changes under different influences called forces.
\emph{The acceleration of an object is directly proportional to the net force acting on it and inversely proportional to its mass.}
$$\vec{a} = \frac{\sum \vec{F}}{m} \mathrm{, or } \sum{\vec{F}} = \vec{F}_\mathrm{net} = m \vec{a}$$
\underline{Newton's second law is applicable on every object.}

\subsection{Unit of Force}
Newton's first and second laws allow us to define force more precisely.

\hspace{1mm}

\noindent
A \textbf{force} is an external influence on an object that causes it to accelerate relative to an inertial reference frame. The direction of the force is that of the acceleration it causes and the magnitude is the product of the mass of the object and the magnitude of the acceleration.
%intrinsic prop

\hspace{1mm}

\noindent
The SI unit of force is the newton, or N. 1 N = 1 kg m/s$^2$. In the U.S. customary system, the unit of force is the pound. 1 N = 0.225 lb. The units of mass and acceleration in the U.S. customary system are the slug and ft/s$^2$.

\subsection{Newton's Third Law}
\emph{Forces always occur in equal and opposite pairs. If object A exerts a force $\vec{F}_{A, B}$ on object B, an equal but opposite force $\vec{F}_{B, A}$ is exerted by object B on object A.} $$\vec{F}_{A, B} = -\vec{F}_{B, A}$$
The pair of forces are parts of an interaction between two objects. One force is called action and the other reaction.
\\It is important to note that action and reaction forces act on different objects.

\hspace{1mm}

\noindent
\textit{Newton's first, second, and third law statements are valid only in inertial reference frames.}

\section{Forces in Nature}
\textit{Identify the action and reaction of each force.}

\hspace{1mm}

\noindent
Forces result from the physical contact between two objects are called \textbf{contact forces}. Examples include a pull on a spring, a push on a wall, or a kick on a football.
\\There is another class of forces that doesn't involve any direct physical contact. They act at a distance. They are called \textbf{field forces}. All known fundamental forces are field forces:
\begin{itemize}
	\item The gravitational force---the force of mutual attraction between objects.
	\item The electromagnetic force---the force between electric charges.
	\item The strong nuclear force---the force between subatomic particles.
	\item The weak nuclear force---the force between subatomic particles during certain radioactive decay processes.
\end{itemize}
It is conceptually hard to imagine a force that acts at a distance so the concept of a field is introduced. Thus, an object of mass $M$, such as the Sun, creates and invisible influence that stretches throughout space. A second object of mass $m$, such as Earth, interacts with the \textit{field} of the Sun, not directly with the Sun itself.

\hspace{1mm}

\subsection{Weight}The force due to gravity.
\\When air resistance is neglected, all falling objects near Earth's surface have the same acceleration (by experiment): $$g=9.81 \mathrm{ m/s} ^2 \approx 10 \mathrm{ m/s} ^2$$
The force causing this acceleration is the gravitational force on the object, called weight. If the weight is the only force acting on the object, the object is said to be in \textbf{free-fall}. By Newton's second law, the wight of any object of mass $m$ is $$\vec{w} = m \vec{g}$$
By Newton's law of universal gravitation, the magnitude of the gravitational force $\vec{F}$ on an object above Earth's surface at an arbitrary distance $r$ from the center is $$F = \frac{GmM_E}{r^2}$$
Both $\vec{F}$ and $\vec{g}$ point to the center of the Earth. The vector $\vec{g}$ is the force per unit mass exerted by the earth on any object and is called the \textbf{gravitational field} of the earth.

%\hspace{1mm}

\noindent
\textbf{Apparent Weight} When you stand on a spring scale, your feet feel the force exerted by the scale. The scale is calibrated to read the force it must exert (by the compression of its springs) to balance your weight. This balance force is called your \textbf{apparant weight}. If there is no force to balance your weight, as in free-fall, your apparent weight is zero. This condition, called \textbf{weightlessness}, is experienced by astronauts in orbiting satellites.

\hspace{1mm}

\begin{wrapfigure}{r}{0.2\textwidth}
\begin{center}
\begin{asy}
size(4cm);
draw((0,0)--(100,0));
draw((10,0)--(10,80));
draw((10,80)--(90,80));
draw((90,80)--(90,0));
dot((50,40));
draw((50,40)--(50,-10),EndArrow);
label("$m \vec{g}$",(50,40)--(50,-10),E);
dot((50,-70));
draw((50,-70)--(50,-20),EndArrow);
label("$m \vec{g'}$",(50,-70)--(50,-20),E);
draw((20,75)--(20,0),EndArrow);
label("$\vec{F}_{n}^{'}$",(20,60),E);
draw((20,-75)--(20,0),EndArrow);
label("$\vec{F}_{n}$",(20,-75)--(20,0),W);
\end{asy}
\end{center}
\end{wrapfigure}

\noindent
\textbf{Difference between Mass and Weight}

\noindent
Weight, unlike mass, is not an intrinsic property of an object. An object on Earth and on the moon have the same mass, would it weight the same?


\hspace{1mm}

\noindent
Weight on Earth is the gravitational pull by the Earth. In outer space you would be weightless.


\hspace{1mm}

\noindent
Mass is the intrinsic property of an object that measures its inertial resistance to acceleration. It is the amount of matter in the object. Mass does not depend on the location in the universe. You do not become massless in outer space.

\subsection{Contact Forces}
Contact forces are electromagnetic in origin. They are exerted between the surface molecules in contact.

\hspace{1mm}

\noindent
\textbf{Normal Force} When a solid surface is pushed or compressed against, it pushes back with a force perpendicular to the contacting surface. This force is called \textbf{normal force}.

\hspace{1mm}

\noindent
\textbf{Tension} The magnitude of the force that one segment of a string or rope exerts on an adjacent segment.


\hspace{1mm}

\noindent
The tension can vary throughout the rope. However, for most problems, the masses of strings or ropes are assumed to be so small that variations in the tension due to their weight or acceleration can be neglected.


\hspace{1mm}

\noindent
This can be easily seen by application of Newton's laws to a segment of a rope.

\hspace{1mm}

\noindent
\textbf{Spring Force} When a spring is compressed or extended by a small amount $\Delta{x}$, the force it exerts is found experimentally to be given by Hooke's law $$F_x = -k \Delta{x}$$

\begin{wrapfigure}{r}{0.3\textwidth}
\begin{center}
\includegraphics[width=4cm]{images/04_!friction.JPG}
\end{center}
\end{wrapfigure}

\subsection{Friction} Surfaces in contact can also exert forces on each other that are parrallel to the contacting surfaces.

\hspace{1mm}

\noindent
A force that opposes the motion or attempted motion of an object past another object or through a fluid is called a \textbf{frictional force}.

\hspace{1mm}

\noindent
Friction is what enables us to walk and drive a car.

\hspace{1mm}

\noindent
\textbf{Static Friction} The resistive force that opposes the attempted motion of an object past another object with which it is in contact.

\hspace{1mm}

\noindent
It is between the surfaces of the two obejcts and it has a maximum. Data show the maximum static friction is proportional to the normal force $F_n$ exerted by one surface on the other: $$f_s \leq \mu_s F_n = f_{s,max}$$ $\mu_s$ is the coefficient of static friction. it depends on the nature of the surface in contact.


\hspace{1mm}

\noindent
If you exert a horizontal force smaller than $f_{s,max}$ on a box, the frictional force will just balance this horizontal force.

\hspace{1mm}

\begin{wrapfigure}{r}{0.4\textwidth}
\begin{center}
\vspace{-20pt}
\begin{asy}
import graph;
xaxis("$F_{app}$", xmin=0);
yaxis("$f$", ymin=0);
labelx("Applied force",(45,0));
axes(ylabel="Frictional force\ \ \ \ \ ");
size(6cm);
draw((0,0)--(50,50));
draw((50,50)--(55,40));
draw((55,40)--(100,40));
draw((30,10)--(10,10),EndArrow);
label("$f_{s} = F_{app}$",(30,10),E);
draw((50,70)--(50,50),EndArrow);
label("$f_{s, max} = \mu_{s} F_{n}$",(50,70),N);
draw((75,30)--(75,40),EndArrow);
label("$f_{k} = \mu_{k} F_{n}$",(75,30),S);
\end{asy}
\end{center}
\end{wrapfigure}

\noindent
\textbf{Kinetic Friction} The resistive force that opposes the motion of an object past another object with which it is in contact. It is between the surfaces of the two objects.
\\The motion has started.
\\Data show $$f_k = \mu_k F_n$$ $\mu_k$ is the coefficient of kinetic friction. It depends on the nature of the surface in contact.
\\It does not depend on velocity, so it is constant once the motion starts.
\\Experimentally, $\mu_k < \mu_s$.

\hspace{1mm}

\noindent
\textbf{Friction Explained} Friction is a complex, incompletely understood phenomenon that arises from the attraction of molecules between two surfaces that are in close contact. This attraction is of electromagnetic nature. It is short ranged and becomes negligible at distances of only a few atomic diameters.

\subsection{Drag Forces}
The resistive force that opposes the motion of an object through a fluid, and the speed of the object relative to the fluid such as air or water.
\\Drag force depends on the shape of the object, the properties of the fluid, and the speed of the object relative to the fluid.
\\Data show $$f = b v^n$$ where the coefficient $b$ depends on object's size and shape and the fluid, $n = 1$ at low speeds and $n = 2$ at higher speeds.

\section{Application of Newton's laws}

\subsection{Free-Body Diagram}
A force diagram that shows schematically all the forces acting on a system
\\Isolate, or free, the object in question from everything else and (usually) represent it by a dot.
\\Draw all the forces acting on the object. (Apply Newton's third law when necessary.)
\\Label each vector to indicate what type of force it represents, for example, $W$ or $mg$ for weight, $F_n$ for normal force, $f$ for friction, $T$ for tension, etc.

\hspace{1mm}

\fbox{\parbox{6.25in}{
\underline{Strategy:}
\\\textit{Read} the question carefully
\\\textit{Draw} a diagram of the situation, a free-body diagram for each body involved, and a coordinate system; label the forces in conventional notations, which defines the symbols.
\\\textit{Apply} Newton's second law in the components form: $\sum F_x = m a_x$ and $\sum F_y = m a_y$
\\\textit{Solve} the set of equations.
\\\textit{Check} for the validity and dimensions of the results.
}}

\subsection{Objects in Equilibrium} Objects that are at rest or moving with constant velocity are said to be in equilibrium. $$\sum F = 0$$ or in components form: $\sum F_x = 0$ and $\sum F_y = 0$.

\subsection{Air Drag and Terminal Speed} Air resistance isn't always undesirable.
Two terminal speeds are usually reached by a sky diver: one before release of the parachute $\left(\sim 200 \mathrm{ km/h}\right)$ and one after $\left(\sim 20 \mathrm{ km/h}\right)$.
\\When air drag is not negligible, if a heavy and light object have the same shape, which one falls to the ground faster?

\section{Motion Along a Curved Path}
A particle moving with constant speed $v$ along a circle of radius $r$ has centripetal acceleration $$a_c = v^2/r$$
A particle moving with \textit{varying} speed $v$ along a circle of radius $r$ has centripetal acceleration $a_c = v^2/r$ and a tangential acceleration $a_t = \lim{\Delta{t} \to 0} (\Delta{v}/\Delta{t}) = dv/dt$.

\hspace{1mm}

\noindent
For general motion along a curve, we can treat a portion of the curve as an arc of a circle. The particle then has acceleration $a_c = v^2/r$ towards the center of curvature and if the speed is changing, it has tangential acceleration $a_t = \lim{\Delta{t} \to 0} \Delta{v}/\Delta{t}) = dv/dt$


\hspace{1mm}

\noindent
The net force is in the direction of the acceleration. The component of the net force in the centripetal direction is called the centripetal force. It is always directed inward - towards the center of curvature of the path. (It is not a new force, just a name for the component of the net force.)
\end{document}