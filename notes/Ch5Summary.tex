\documentclass[11pt,a4paper]{report}
\usepackage{asymptote}
\usepackage{wrapfig}
\addtolength{\oddsidemargin}{-.875in}
\addtolength{\evensidemargin}{-.875in}
\addtolength{\textwidth}{1.75in}
\addtolength{\topmargin}{-.875in}
\addtolength{\textheight}{1.75in}
\begin{document}
\pagestyle{empty}
\setcounter{secnumdepth}{0}

\begin{center}
\Large{CHAPTER 5 SUMMARY. \textbf{Energy}}

\large{Justin Yang}

\sc{November 22, 2012}
\end{center}

\begin{wrapfigure}{r}{0.2\textwidth}
\vspace{-20pt}
\begin{center}
\includegraphics[width=0.18\textwidth]{images/05_!1work.JPG}
\end{center}
\end{wrapfigure}
\section{Work}
The \textbf{work} done by a constant force $\vec{F}$ that moves an object a displacement $\Delta{x}$ is defined as $$W \equiv F \Delta{x} \cos{\theta}.$$ So $W = \vec{F} \cdot \Delta{\vec{x}}$.

\begin{wrapfigure}{r}{0.2\textwidth}
\vspace{-10pt}
\begin{center}
\includegraphics[width=0.18\textwidth]{images/05_!3varying.JPG}
\end{center}
\end{wrapfigure}

\noindent
Work is a scalar. The SI unit of work is the \textit{joule} (J), $1 \mathrm{\ J} = 1 \mathrm{\ N} \cdot \mathrm{m} = 1 \mathrm{\ kg} \cdot \mathrm{m}^2 / \mathrm{s}^2$.

\smallskip

\noindent
A \textbf{particle} is any object where all of its parts undergo equal $\Delta{x}$ over any $\Delta{t}$.
\\The total work done on a particle is the same as the work done by the net force on the particle, so the work done is the area under the $F_x$-versus-$x$ curve: $$W = \sum{\vec{F_i}} \cdot \Delta{\vec{x}} = \vec{F}_\mathrm{net} \cdot \Delta{\vec{x}}.$$

\section{Kinetic Energy}
Under a constant \textit{net} force $F_\mathrm{net}$ acting along a straight line on a particle of mass $m$, which is displaced by $\Delta{x}$ along the straight line, the work done on the particle is $$W = F_\mathrm{net} \Delta{x}.$$
Applying Netwon's second law $F_\mathrm{net} = ma$ and the kinematic relation $v^2 - v_0^2 = 2a \Delta{x}$, we have $$W = F_\mathrm{net} \Delta{x} = ma \Delta{x} = \frac{1}{2} mv^2 - \frac{1}{2} mv_0^2.$$
The quantity $\frac{1}{2} mv^2$ is defined as the \textbf{kinetic energy} of the particle $$K \equiv \frac{1}{2} mv^2.$$
\\Kinetic energy is a scalar. The SI unit of kinetic energy is the same as work: $\mathrm{kg} \cdot \mathrm{m}^2 / \mathrm{s}^2$ or J.
\\Kinetic energy depends on the mass and speed of the particle but not the direction of motion.

\smallskip

\noindent
$W = \Delta{K}$. This is true even when the force is varying. This is known as the \textbf{work-energy theorem}.

\section{Potential Energy}
The \textbf{potential energy} of a system is the energy associated with the configuration of the system. Often the work done by external forces on a system may result in an increase in the potential energy of the system.

\begin{wrapfigure}{r}{0.1\textwidth}
\vspace{-23pt}
\begin{center}
\includegraphics[width=0.09\textwidth]{images/05_!4gpe.JPG}
\end{center}
\vspace{-20pt}
\end{wrapfigure}

\smallskip

\noindent
\textit{Gravitational Potential Energy} The gravitational force between an object of mass $m$ and the Earth is $\vec{F} = -mg\,\hat{j}$, where $h$, $h_0 \ll r_E$, so the work done by gravity is $$W_g = \vec{F} \cdot \Delta{x} = -mg\,\hat{j} \cdot \Delta{\vec{x}} = -mg \Delta{h} = -mg \left(h - h_0\right).$$
When the object is near the surface of the Earth, the gravitational potential energy $$U_g \equiv mgh.$$
Thus, the work done by gravity is at the expense of the gravitational potential energy: $$W_g \equiv -\Delta{U_g}.$$
\textit{Potential Energy of a Spring} The work done by the spring force, $F = -kx$, is given as $$W_s = -\frac{1}{2} \left(kx_1 + kx_2\right)\left(x_2 - x_1\right) = -\left(\frac{1}{2} kx_2^2 - \frac{1}{2} kx_1^2\right).$$
When the spring potential energy is zero at $x = 0$, the spring potential energy can be defined as $$U_s \equiv \frac{1}{2} kx^2.$$
The work done by the spring force is then at the expense of the spring potential energy $$W_s = -\Delta{U_s}.$$

\subsection{Conservative Force and Potential-Energy Function}

\begin{wrapfigure}{r}{0.2\textwidth}
\vspace{-20pt}
\begin{center}
\includegraphics[width=0.18\textwidth]{images/05_!5conservative.JPG}
\end{center}
\vspace{-20pt}
\end{wrapfigure}

A force is conservative if on a particle $W_\mathrm{net} = 0$ around \textit{any} closed path.
\\We can use this property to define a \textbf{potential-energy function} $U$ such that the force is the negative of the slope of the potential-energy $U$-versus-$x$ curve: $$W = \sum_i \vec{F} \cdot \Delta{\vec{x}_i} = -\Delta{U}.$$
\textbf{Non-conservative forces} are forces that are not conservative.

\section{Conservation of Mechanical Energy}
A \textbf{system} is a collection of particles. All forces are either \textbf{external} or \textbf{internal}. The change in $E_\mathrm{net}$ of a system is done through work and heat. Since $K = \sum{K_i}$, we obtain by the work-energy theorem $$W_\mathrm{net} = \sum{\Delta{K_i}} = \Delta{K} = W_\mathrm{ext} + W_\mathrm{nc} + W_\mathrm{c}.$$
The work done by all internal conservative forces can be recast as the change in the total potential energy of the system: $$W_\mathrm{c} = -\Delta{U}.$$
The sum $E_\mathrm{mech} = K + U$ is known as the total mechanical energy of the system, $$W_\mathrm{ext} + W_\mathrm{nc} = \Delta{K} + \Delta{U} = \Delta{\left(K + U\right)} = \Delta{E_\mathrm{mech}}.$$
When $W_\mathrm{ext} = 0$ and $W_\mathrm{nc} = 0$, we get the \textbf{conservation of mechanical energy}: $$K_f + U_f = K_i + U_i.$$

\section{Conservation of Energy}
For an isolated system, we have $W_\mathrm{ext} = 0$ and we may account of $W_\mathrm{nc}$ by changes in forms of energy other than mechanical energy. \textbf{The law of energy conservation:} $$E = E_\mathrm{mech} + E_\mathrm{therm} + E_\mathrm{chem} + E_\mathrm{other}.$$
Work and heat are the ways to transfer energy in or out of a system. When $\Delta{Q} = 0$, we have: $$W_\mathrm{ext} = \Delta{E} = \Delta{E_\mathrm{mech}} + \Delta{E_\mathrm{therm}} + \Delta{E_\mathrm{chem}} + \Delta{E_\mathrm{other}}.$$

\section{Power}
Power is the rate at which energy is transferred. The average \textbf{power} supplied by a force $\vec{F}$ is the rate at which the force does work: $$\bar{P} = \frac{\Delta{W}}{\Delta{t}} = \vec{F} \cdot \frac{\Delta{\vec{x}}}{\Delta{t}} = \vec{F} \cdot \vec{v}_{av},$$ $$P = \lim_{\Delta{t} \to 0} \vec{F} \cdot \vec{v}.$$
The SI unit of power is J/s, also called the \textbf{watt}. $1 \mathrm{\ W} = 1 \mathrm{\ J} / \mathrm{s} = 1 \mathrm{\ kg} \cdot \mathrm{m}^2 / \mathrm{s}^3$.
\end{document}