\documentclass[11pt,a4paper]{report}
\usepackage[margin=1in]{geometry}

\begin{document}
\pagestyle{empty}
\begin{center}
\Large{CHAPTER 2 SUMMARY. \textbf{Motion in One Dimension}}

\large{Justin Yang}

October 16, 2012
\end{center}

\textbf{Kinematics} Description of motion

\section*{Displacement, Velocity, and Speed}

An object moving from intial position $x_{i}$ to final position $x_{f}$ has \textbf{displacement} $$\Delta{x}=x_{f}-x_{i}.$$

\noindent
The \textbf{average velocity} of the object is the ratio of the displacement to the time it takes for the displacement $\Delta{t}=t_{f}-t_{i}$,
$$v_{av}=\frac{\Delta{x}}{\Delta{t}}.$$

\noindent
The \textbf{average speed} of the object is the ratio of the distance traveled to the time it takes, $$\bar{v}=\frac{\Delta{s}}{\Delta{t}}.$$

\noindent
The average velocity and the average speed of an object are very different.

\noindent
\textit{Geometric Interpretation}: The average velocity is the slope of the straight line connecting the points $\left(t_{1}, x_{1}\right)$ and $\left(t_{2}, x_{2}\right)$ in the $x$-versus-$t$ plot.

\noindent
\textbf{Instantaneous velocity} is defined as $$v\left(t\right)=\lim_{\Delta{t} \to 0} \frac{\Delta{x}}{\Delta{t}}.$$

\noindent
The \textbf{instantaneous speed} is the magnitude of the instantaneous velocity.

\section*{Acceleration}

The rate of change of the instantaneous velocity with respect to time.

The \textbf{average acceleration} of an object is the ratio of the \textit{change} in velocity to the time it takes for the change $\Delta{t}=t_{f}-t_{i}$, $$a_{av}=\frac{\Delta{v}}{\Delta{t}}.$$

The \textbf{instantaneous acceleration} is the slope of the line tangent to the $v$-versus-$t$ curve. $$a=\lim_{\Delta{t} \to 0} \frac{\Delta{v}}{\Delta{t}}.$$

\subsection*{Motion with Constant Acceleration}

The motion of a particle that has constant acceleration is common in nature. When air resistance is negligeble, the free fall of an object near Earth's surface has acceleration $g=9.81 m/s^{2}$, which Galileo was the first to conclude.

\noindent
We can use the ``Big Five'' under constant acceleration:
$$v=v_{0}+at$$
$$\Delta{x}=v_{0}t+\frac{1}{2}at^{2}$$
$$v^{2}-v^{2}_{0}=2a\Delta{x}$$
$$\Delta{x}=\frac{1}{2}\left(v_{0}+v\right)t$$
$$\Delta{x}=vt-\frac{1}{2}at^{2}$$
\end{document}